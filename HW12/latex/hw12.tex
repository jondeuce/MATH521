%% ------------------------------------------------------ %%
%% ------------ Additional Packages/Commands ------------ %%
%% ------------------------------------------------------ %%
\usepackage[T1]{fontenc}
\usepackage[utf8]{inputenc}
\usepackage{newunicodechar}
\usepackage[margin=0.6in,heightrounded]{geometry}
\usepackage{textcomp}
\usepackage{gensymb}
\usepackage{amssymb}
\usepackage{amsmath}
\usepackage{amsfonts}
\usepackage{mathtools}
\usepackage{mathrsfs}
\usepackage{eucal}
\usepackage{units}
\usepackage{siunitx}
\usepackage{color}
\usepackage{bm} % for bold vectors
\usepackage{upgreek} % for bold greek-letter vectors
%\usepackage{algorithmicx}
\usepackage{algorithm,algorithmic,setspace} % http://ctan.org/pkg/algorithms
%\usepackage[noend]{algpseudocode} % http://ctan.org/pkg/algorithmicx
\usepackage[section]{placeins}
\usepackage{afterpage} % for clearing page after figure
\usepackage{titling} % for title
\usepackage{caption} % for caption in minipage environment
\usepackage{authblk} % for affiliations
\usepackage{parskip} % Removes indentation from paragraphs
%\usepackage[bottom]{footmisc}
%\raggedbottom % allow paragraphs to end before page bottom
%\usepackage{graphicx}
%\usepackage{textpos} %for positioning logo
%\usepackage{animate} %for animated gif
\usepackage{xcolor,xspace,soul} % Colour, spacing, and highlighting
\usepackage{calc}

% Spacing
\setlength{\footskip}{\paperheight-(1.0in+\voffset+\topmargin+\headheight+\headsep+\textheight)-0.3in}
\AtBeginDocument{%
 \abovedisplayskip=8pt
 \abovedisplayshortskip=8pt
 \belowdisplayskip=8pt
 \belowdisplayshortskip=8pt
}

% ************************************************
% ***************** NEW COMMANDS *****************
% ************************************************

%\renewcommand{\labelenumi}{(\alph{enumi})} % Use letters for enumerate
\let\vaccent=\v % rename builtin command \v{} to \vaccent{}

%% MRI Specific
\newcommand{\conj}[1]{\overline{#1}}
%\newcommand{\Mxy}{\widetilde{M}_{xy}}
%\newcommand{\Mxy}{\widetilde{M}}
\newcommand{\Mxy}{\mathcal{M}}
\newcommand{\CDecay}{\Gamma}
\newcommand{\MM}{\ensuremath{\v{M}}}
\newcommand{\Mt}{\ensuremath{\v{M}_{\bot}}}
\newcommand{\BB}{\ensuremath{\v{B}}}
\newcommand{\Bo}{\ensuremath{\v{B_0}}}
\newcommand{\rr}{\ensuremath{R(\v{x})}}
\newcommand{\ww}{\ensuremath{\omega(\v{x})}}
\newcommand{\uu}{\ensuremath{\v{u}}}


%% MISC
\newcommand{\ab}[1]{\left| #1 \right|} % for absolute value
\newcommand{\avg}[1]{\left< #1 \right>} % for average
\let\underdot=\d % rename builtin command \d{} to \underdot{}
\let\baraccent=\= % rename builtin command \= to \baraccent
\renewcommand{\=}[1]{\stackrel{#1}{=}} % for putting numbers above =
\providecommand{\fr}{\frac}
\providecommand{\RR}{\mathbb{R}}
\providecommand{\CC}{\mathbb{C}}
\providecommand{\NN}{\mathbb{N}}
\providecommand{\e}{\epsilon}
\DeclareMathOperator{\di}{d\!}
\DeclareMathOperator{\Arg}{Arg}
\newcommand*\ieval[3]{\left.#1\right\rvert_{#2}^{#3}}

%% Vectors
\renewcommand{\v}[1]{\ensuremath{\mathbf{#1}}} 
\newcommand{\gv}[1]{\ensuremath{\mbox{\boldmath$ #1 $}}} % for vectors of Greek letters
\newcommand{\uv}[1]{\ensuremath{\mathbf{\hat{#1}}}} % for unit vector
\providecommand{\wave}[1]{\v{\tilde{#1}}}

%% DERIVATIVES
\renewcommand{\d}[2]{\frac{d #1}{d #2}} % for derivatives
\newcommand{\dubd}[2]{\frac{d^2 #1}{d #2^2}} % for double derivatives
\newcommand{\pd}[2]{\frac{\partial #1}{\partial #2}} % for partial derivatives
\newcommand{\ddt}[1]{\ensuremath{{#1}_t}}
\newcommand{\pdd}[2]{\frac{\partial^2 #1}{\partial #2^2}} % for double partial derivatives
\newcommand{\dx}[1]{\ensuremath{\operatorname{d}\!{#1}}}

%% Operators
\newcommand{\Gradient}{\ensuremath{\mbox{$\nabla$}}} %\newcommand{\Gradient}{\ensuremath{\mbox{\boldmath$\nabla$}}} % gradient
\newcommand{\Laplacian}{\ensuremath{\mbox{$\Delta$}}} % laplacian
%\newcommand{\Lap}{\nabla^2}
%\newcommand{\conv}{\ensuremath{\mbox{\ast}}} % convolution
\newcommand{\conv}{\ast} % convolution

%% Norms
\newcommand{\Norm}[1]{\ensuremath{\left\lVert #1 \right\rVert}} % generic norm
\newcommand{\ANorm}[2]{\ensuremath{\left\lVert #1 \right\rVert_{#2}}} % norm with respect to specified A
\newcommand{\LTwoNorm}[1]{\ensuremath{\left\lVert #1 \right\rVert_{L^2}}} % L2 norm
\newcommand{\HOneNorm}[1]{\ensuremath{\left\lVert #1 \right\rVert_{H^1}}} % H1 norm

%% Math Operators
\DeclareMathOperator{\spann}{span}

% ***************************************************
% ******************* TEXT AND CODE *****************
% ***************************************************

%% Text
\newcommand{\mathcolorbox}[2]{\colorbox{#1}{$\displaystyle #2$}}
\newcommand{\hlfancy}[3]{\textcolor{#1}{\sethlcolor{#2}\hl{#3}}}
\newcommand{\TODO}[1]{\hlfancy{red}{yellow}{\textbf{TODO: #1}}}

%% Code
%\newcommand{\code}[1]{\mintinline{C}{#1}}
%\newcommand{\code}[1]{\texttt{#1}}
%\newcommand{\code}[1]{\lstinline[columns=fixed]{#1}}
%\newcommand{\includecode}[1]{\lstinputlisting{#1}}

% ***************************************************
% **************** SPIN-ECHO COMMANDS ***************
% ***************************************************

% Remove algorithm number label, as there is only one
%\captionsetup[algorithm]{labelformat=empty}

%\newcommand{\vect[1]}{\vec{#1}} %vectors using arrows
%\newcommand{\vect}[1]{\mathbf{#1}} %vectors using boldface
%\newcommand{\bvect}[1]{\bm{#1}} % \mathbf doesn't work for greek/mathcal letters
%\newcommand{\bvect}[1]{#1} % looks better without bold

\newcommand{\matr}[1]{\mathbf{#1}} % matrix: undergraduate algebra version
%\newcommand{\matr}[1]{\bm{#1}}    % matrix: ISO complying version

%\newcommand{\tensor}[1]{\overleftrightarrow{\matr{#1}}} % for diffusion tensor
%\newcommand{\tensor}[1]{\overset{\text{\tiny$\leftrightarrow$}}{\matr{#1}}}
\newcommand{\tensor}[1]{\matr{#1}} % for diffusion tensor
%\def\tensor#1{\protect\@ontopof{#1}{\leftrightarrow}{1.15}\mathord{\box2}}

% for capital 'M' for molar
%\DeclareSIUnit\Molar{\text{M}}

%% Prevent inline equations from breaking over lines
%\relpenalty=9999 %10000 means never, 9999 means almost never
%\binoppenalty=9999
%
%% Prevent over hyphenation of words across lines
%\pretolerance=2000 %10000 means never, 9999 means almost never
%\tolerance=2000 
%\emergencystretch=10pt % maximum width of allowed extra space is 10pt

%% Begin added or changed by Alex Rauscher
\renewcommand{\floatpagefraction}{0.8} % so that pictures don't occupy an entire column

% force figures to be at the top of the page, not float in the middle
\makeatletter
\setlength{\@fptop}{0pt}
\makeatother

%\DeclareUnicodeCharacter{01E7}{ǧ} % does NOT work; pdflatex halts
%\newunicodechar{ǧ}{\u{g}} % not sure why, but this needs to be defined

% For a regular tilde...
\newcommand{\mytilde}{\raise.17ex\hbox{$\scriptstyle\mathtt{\sim}$}}
\newcommand{\myast}{\lower.17ex\hbox{*}}

%\documentclass[10pt,letterpaper]{scrartcl}
%\usepackage{amsfonts,amsmath,amssymb,braket,xcolor,dsfont,enumerate,fontawesome,graphicx}
\usepackage{amsfonts,amsmath,amssymb,braket,xcolor,enumerate,graphicx}
\usepackage[hidelinks]{hyperref}
\usepackage{listings,multicol,mathtools,textcomp,tikz,pgfplots,wrapfig}
\usepackage[inner=2cm,outer=2cm,top=2cm,bottom=2cm]{geometry}
\usepackage{tabularx}
\usepackage{booktabs}
\usepackage{appendix}
\usepackage{fancyhdr}
\usetikzlibrary{arrows}
%\pgfplotsset{compat=1.12}

\pagestyle{empty}
\setlength{\parindent}{0pt}
\setlength{\parskip}{6pt}

\newcommand{\dx}{\;\mathrm{d}x}
\newcommand{\ds}{\;\mathrm{d}s}

\begin{document}

\begin{minipage}{.2\textwidth}
\includegraphics[width=42pt]{ubc-logo.png}
\end{minipage}
\hfill
\begin{minipage}{.75\textwidth}
\setlength{\parskip}{6pt}
\begin{flushright}
{\sffamily
\textbf{MATH521}\\
\textbf{Numerical Analysis of Partial Differential Equations}

Winter 2017/18, Term 2\\
Timm Treskatis
}
\end{flushright}
\end{minipage}

\section*{Homework Assignment 12}

%Please submit the following files as indicated below: \hfill \faFileCodeO \: source code \hfill \faFilePdfO \: PDF file \hfill \faFilePictureO \: image file \hfill \faFileMovieO \: video file

\paragraph*{Question 1 $\vert$ 3 marks}% $\vert$ \faFileCodeO{} \: \faFilePdfO}

Today we're solving the linear advection equation
\begin{subequations}\label{eq:advection}
\begin{align}
\frac{\partial u(t)}{\partial t} + \operatorname{div} (u(t)a(t)) &= 0 && \text{in } Q = ]0,T[ \times \Omega\\
u(0) &= u_0 && \text{in } \Omega\\
u(t) &= g(t) && \text{on } \Sigma_- = \Set{(t,x) \in \: ]0,T[ \times \partial\Omega | a(t,x)\cdot n(x) < 0},
\end{align}
\end{subequations}
where $a: \: ]0,T[ \times \Omega \to \mathbb{R}^2$ is a given vector field, thought of as the flow velocity of a carrier fluid, and $u_0: \Omega \to [0,1]$ the initial concentration of a solute. $g: \: ]0,T[ \times \partial\Omega \to [0,1]$ prescribes the concentration on that part of the boundary $\partial \Omega$ where the flow of solvent is directed into the domain and we use $n$ to denote the outward pointing unit normal vectors on $\partial \Omega$. For simplicity we assume in what follows that $a$ and $g$ do not depend on time.

Recall that upwind discontinuous \textsc{Galerkin} methods employ the bilinear form
\begin{equation*}
\sum_{T\in\mathcal{T}^h} -\int\limits_T ua \cdot \nabla v \dx + \sum_{e\in\mathcal{E}^h} \int\limits_e [v] u_{\text{up}} a \cdot n_+ \ds
\end{equation*}
to discretise the transport term $\operatorname{div} (ua)$.
\begin{enumerate}[(a)]
\item We define the positive and negative parts of a function $f$ by
\begin{equation*}
f^{\text{pos}} = \frac{f + \lvert f \rvert}{2} \qquad \text{and} \qquad f^{\text{neg}} = \frac{f - \lvert f \rvert}{2}.
\end{equation*}
Re-write the edge integral in the upwind DG form in terms of $g$, $u_+$, $u_-$, $(a\cdot n_+)^{\text{pos}}$, $(a\cdot n_+)^{\text{neg}}$ and $[v]$:
\begin{solution}
There are two cases:
\begin{itemize}
\item If $e$ is an interior edge:
\begin{align*}
\int\limits_e [v] u_{\text{up}} a \cdot n_+ \ds
&= \int\limits_e [v] u_{\text{up}} \left[ (a\cdot n_+)^{\text{pos}} + (a\cdot n_+)^{\text{neg}} \right] \ds \\
&= \int\limits_e [v] u_+ (a\cdot n_+)^{\text{pos}} \ds + \int\limits_e [v] u_- (a\cdot n_+)^{\text{neg}} \ds
\end{align*}
The second line follows from the fact that $(a\cdot n_+)^{\text{pos}} = 0$ when $a\cdot n_+ < 0$, and therefore $u_{\text{up}} = u_+$ wherever the integrand in non-zero. Similarly, $u_{\text{up}} = u_-$ in the second term when $(a\cdot n_+)^{\text{neg}}$ is non-zero.
\vfill
\item If $e$ is an exterior edge (no $\pm$ subscripts or $[\:]$ brackets needed, there is only one neighbouring triangle):
\begin{align*}
\int\limits_e [v] u_{\text{up}} a \cdot n_+ \ds
&= \int\limits_e v u_{\text{up}} \left[ (a\cdot n)^{\text{pos}} + (a\cdot n)^{\text{neg}} \right] \ds \\
&= \int\limits_e v u (a\cdot n)^{\text{pos}} \ds + \int\limits_e v g (a\cdot n)^{\text{neg}} \ds
\end{align*}
Where we have used the fact that $u=g$ when $a\cdot n < 0$.
\end{itemize}
\end{solution}

\newpage

\item Complete the \textsf{FEniCS} script \texttt{hw12.py} to solve the linear advection equation with
\begin{equation*}
T = 2\pi \qquad \qquad a(x) = \left(
\begin{array}{c}
-x_2\\ x_1
\end{array}
\right) \textcolor{gray}{ = r \left(
\begin{array}{c}
-\sin \phi \\ \cos \phi
\end{array}
\right)  \text{ in polar coordinates}}  \qquad \qquad  g(x) \equiv 0
\end{equation*}
and the domain and initial data provided. Note that with these data, the advection equation rotates the initial field $u_0$ around the origin in anticlockwise direction, like a rigid body on a turntable.

Use a moderate degree (e.g. $r=1$ or $2$) for the DG-discretisation in space and use the method we are most familiar with, the $\theta$-method, for time stepping. Any value of $\theta$ which results in a stable scheme is fine.

\emph{Hint:} In \textsf{FEniCS} we use \verb|dS| for integrals over interior edges and \verb|ds| for integrals over exterior edges. You must add \verb|('+')| or \verb|('-')| to all discontinuous functions in integral expressions containing \verb|dS|, e.g.
\begin{verbatim}
u('-')*jump(v)*dS
\end{verbatim}
to specify on what side of the edge they should be evaluated. \verb|jump(v)| is a shortcut for \verb|v('+')-v('-')|.

\item Inspect your numerical solution in \textsf{ParaView}, using the `Warp by Scalar'-Filter. Recall that \textsf{FEniCS} generally exports solution data as continuous, piecewise linear functions, so don't be surprised when you cannot see any discontinuous jumps in your visualisation.

Describe any discrepancies between the numerical solution and the exact solution, using the appropriate terminology.

\emph{Hint: } For general triangular meshes, the upwind DG($r$) method discretises the advection term with an accuracy of order $r+\frac{1}{2}$, which increases to $r+1$ on meshes with certain regularity.

\end{enumerate}

\newpage

\paragraph*{Question 2 $\vert$ 2 marks}% $\vert$ \faFileCodeO{} \:}

Make a copy of your script \texttt{hw12.py} and modify it to solve the advection-diffusion problem
\begin{subequations}\label{eq:advection-diffusion}
\begin{align}
\frac{\partial u(t)}{\partial t} + \operatorname{div} (u(t)a - D\nabla u(t)) &= 0 && \text{in } Q = ]0,T[ \times \Omega\\
u(0) &= u_0 && \text{in } \Omega\\
\left(u(t)a-D\nabla u(t)\right) \cdot n &= 0 && \text{on } \Sigma = ]0,T[ \times \partial\Omega
\end{align}
\end{subequations}
instead, using $D = 0.01$ and all other parameters as in Question 1. The new boundary condition is a no-flux \textsc{Robin} condition. It admits an interpretation of a semi-permeable membrane which allows the solvent, but not the solute to pass through.

Discretise the diffusive flux with the symmetric interior penalty form
\begin{align*}
&\sum_{T\in\mathcal{T}^h} \int\limits_T D\nabla u \cdot \nabla v \dx && \text{(bilinear form of conforming methods)}\\
&- \sum_{e\in\mathcal{E}^h\setminus\mathcal{I}^h} \int_e vD\nabla u \cdot n \ds - \sum_{e\in\mathcal{I}^h} \int_e [v]\left\langle D\nabla u\right\rangle\cdot n_+ \ds && \text{(consistency)}\\
&- \sum_{e\in\mathcal{I}^h} \int_e [u]\left\langle D\nabla v\right\rangle\cdot n_+ \ds && \text{(symmetry)}\\
&+ \sum_{e\in\mathcal{I}^h} \frac{\sigma}{h_e} \int_e [u] [v] \ds && \text{(interior penalty)}
\end{align*}
as derived in class, here written in a form that can be translated directly into \textsf{FEniCS} code (but note that the boundary condition has not been applied yet). Use the penalty parameter $\sigma = 0.1$ and $h_e = \langle h \rangle$.

\emph{Hint:} The commands \verb|h = CellSize(mesh)| and \verb|avg(u)| may be helpful.

\paragraph*{Your Learning Progress}% $\vert$ \faFilePdfO}

What is the one most important thing that you have learnt from this assignment?

\vspace*{3mm}
\hrulefill

\vspace*{3mm}
\hrulefill

Any new discoveries or achievements towards the objectives of your course project?

\vspace*{3mm}
\hrulefill

\vspace*{3mm}
\hrulefill

What is the most substantial new insight that you have gained from this course this week? Any \emph{aha moment}?

\vspace*{3mm}
\hrulefill

\vspace*{3mm}
\hrulefill

\newpage
\begin{appendices}
% ------------------------------------------------------------ %
% Appendix A: Code
% ------------------------------------------------------------ %
\newpage
\lhead{}
\section{}\label{hw12code}
\textbf{\Large\texttt{hw12.py}}
\includecode{../fenics/hw12.py}
\end{appendices}

\end{document}
