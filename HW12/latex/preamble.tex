% Modified Template by Jonathan Doucette and Kevin Multani, original by: Jonathan Ward

\documentclass[10pt,letterpaper]{scrartcl}
%\documentclass[12pt]{article} 
\usepackage[english]{babel}
%\usepackage[utf8]{inputenc}
\usepackage{amsmath} % AMS Math Package
\usepackage{amsthm} % Theorem Formatting
\usepackage{esint} % for contour integrals
\usepackage{amssymb} % Math symbols such as \mathbb
\usepackage{graphicx} % Allows for eps images
\usepackage{multicol} % Allows for multiple columns
\usepackage[inner=2cm,outer=2cm,top=2cm,bottom=2cm]{geometry}
%\usepackage[dvips,letterpaper,margin=1in,bottom=1in]{geometry}
%\usepackage{hyperref}
\usepackage[hidelinks]{hyperref}
\usepackage{parskip} % Removes indentation from paragraphs
\usepackage{xcolor,xspace,soul} % Colour, spacing, and highlighting
\usepackage{mathrsfs}
\usepackage{bm} % For bold math symbols
\usepackage{amscd}
\usepackage[all,cmtip]{xy}
%\usepackage{bbm}
\usepackage{titling}
\usepackage{listing} % for code snippets
%\usepackage{minted} % for code snippets
\usepackage{enumerate}
\usepackage{breqn} % for breaking long equations over lines
\usepackage{fancyhdr}
\usepackage[]{physics}
\usepackage[makeroom]{cancel}
\usepackage{pdfpages}
\usepackage[]{mcode}
\usepackage[title]{appendix}
\usepackage{tikz}

% ***********************************************************
% ********************** BEGIN TITLE PAGE *******************
% ***********************************************************
\newcommand*{\titleGM}{\begingroup % Create the command for including the title page in the document
\hbox{ % Horizontal box
\hspace*{0.2\textwidth} % Whitespace to the left of the title page
\rule{1pt}{\textheight} % Vertical line
\hspace*{0.05\textwidth} % Whitespace between the vertical line and title page text
\parbox[b]{0.75\textwidth}{ % Paragraph box which restricts text to less than the width of the page

{\noindent\Huge\bfseries MATH 521}\\[2\baselineskip] % Title
{\large \textit{Assignment 5}}\\[4\baselineskip]
%{\large \textsc{ Jonathan Doucette }} % Author name

\vspace{0.5\textheight} % Whitespace between the title block and the publisher
{\noindent \today }\\[\baselineskip]
%{\noindent Student Number: 35298124 }\\[\baselineskip] 
%{\noindent \footnotesize All problems below are \textit{Copyright \copyright \space 2018 Timm Treskatis. All Rights Reserved.} }\\[\baselineskip]
} % end parbox
} % end hbox
\endgroup}

 % Sets margins and page size
\pagestyle{fancy} 

%\lhead{Jonathan Doucette}
\rhead{\today}
\rfoot{Page \thepage}
\cfoot{}

\makeatletter % Need for anything that contains an @ command 
\renewcommand{\maketitle} % Redefine maketitle to conserve space
{ \begingroup \vskip 10pt \begin{center} \Huge {\bf \@title}
\vskip 10pt \large \@author \hskip 20pt \@date \end{center}
  \vskip 10pt \endgroup \setcounter{footnote}{0} }
\makeatother % End of region containing @ commands

% ***********************************************************
% ********************** END TITLE PAGE *********************
% ***********************************************************

% ***********************************************************
% ********************** BEGIN NEW COMMANDS *****************
% ***********************************************************

\renewcommand{\labelenumi}{(\alph{enumi})} % Use letters for enumerate
\let\vaccent=\v % rename builtin command \v{} to \vaccent{}

%% MISC
\newcommand{\ab}[1]{\left| #1 \right|} % for absolute value
\newcommand{\avg}[1]{\left< #1 \right>} % for average
\let\underdot=\d % rename builtin command \d{} to \underdot{}
\let\baraccent=\= % rename builtin command \= to \baraccent
\renewcommand{\=}[1]{\stackrel{#1}{=}} % for putting numbers above =
\providecommand{\fr}{\frac}
\providecommand{\RR}{\mathbb{R}}
\providecommand{\CC}{\mathbb{C}}
\providecommand{\NN}{\mathbb{N}}
\providecommand{\e}{\epsilon}
\DeclareMathOperator{\di}{d\!}
\newcommand*\ieval[3]{\left.#1\right\rvert_{#2}^{#3}}

%% Vectors
\renewcommand{\v}[1]{\ensuremath{\mathbf{#1}}} 
\newcommand{\gv}[1]{\ensuremath{\mbox{\boldmath$ #1 $}}} % for vectors of Greek letters
\newcommand{\uv}[1]{\ensuremath{\mathbf{\hat{#1}}}} % for unit vector
\providecommand{\wave}[1]{\v{\tilde{#1}}}

%% DERIVATIVES
\renewcommand{\d}[2]{\frac{d #1}{d #2}} % for derivatives
\newcommand{\dubd}[2]{\frac{d^2 #1}{d #2^2}} % for double derivatives
\newcommand{\pd}[2]{\frac{\partial #1}{\partial #2}} % for partial derivatives
\newcommand{\pdd}[2]{\frac{\partial^2 #1}{\partial #2^2}} % for double partial derivatives
%\newcommand{\dx}[1]{\ensuremath{\operatorname{d}\!{#1}}}
%\newcommand{\dx}{\;\mathrm{d}x}
%\newcommand{\ds}{\;\mathrm{d}s}

%% Operators
\newcommand{\Gradient}{\ensuremath{\mbox{\boldmath$\nabla$}}} % gradient
\newcommand{\Laplacian}{\ensuremath{\mbox{$\Delta$}}} % laplacian

%% Norms
\newcommand{\Norm}[1]{\ensuremath{\lVert #1 \rVert}} % generic norm
\newcommand{\ANorm}[2]{\ensuremath{\lVert #1 \rVert_{#2}}} % norm with respect to specified A
\newcommand{\LTwoNorm}[1]{\ensuremath{\lVert #1 \rVert_{L^2}}} % L2 norm
\newcommand{\HOneNorm}[1]{\ensuremath{\lVert #1 \rVert_{H^1}}} % H1 norm

%% Math Operators
\DeclareMathOperator{\spann}{span}

%% Text
\newcommand{\mathcolorbox}[2]{\colorbox{#1}{$\displaystyle #2$}}
\newcommand{\hlfancy}[3]{\textcolor{#1}{\sethlcolor{#2}\hl{#3}}}
\newcommand{\TODO}[1]{\hlfancy{red}{yellow}{\textbf{TODO: #1}}}

%% Code
%\newcommand{\code}[1]{\mintinline{C}{#1}}
%\newcommand{\code}[1]{\texttt{#1}}
\newcommand{\code}[1]{\lstinline[columns=fixed]{#1}}
\newcommand{\includecode}[1]{\lstinputlisting{#1}}

% ***********************************************************
% ********************** END NEW COMMANDS *******************
% ***********************************************************

% ***********************************************************
% ********************** BEGIN NEW ENVS *********************
% ***********************************************************

% Theorem
\newenvironment{theorem}[2][Theorem]{\begin{trivlist}
\item[\hskip \labelsep {\bfseries #1}\hskip \labelsep {\bfseries #2.}]}{\end{trivlist}}
% Lemma
\newenvironment{lemma}[2][Lemma]{\begin{trivlist}
\item[\hskip \labelsep {\bfseries #1}\hskip \labelsep {\bfseries #2.}]}{\end{trivlist}}
% Corollary
\newenvironment{corollary}[2][Corollary]{\begin{trivlist}
\item[\hskip \labelsep {\bfseries #1}\hskip \labelsep {\bfseries #2.}]}{\end{trivlist}}

% Exercise
\newenvironment{exercise}[2][\textbf{Exercise}]{\begin{trivlist}
\item[\hskip \labelsep {\bfseries #1}\hskip \labelsep {\bfseries #2.}]}{\end{trivlist}}
% Problem
\newenvironment{problem}[2][\textbf{Problem}]{\begin{trivlist}
\item[\hskip \labelsep {\bfseries #1}\hskip \labelsep {\bfseries #2.}]}{\end{trivlist}}
% Question
\newenvironment{question}[2][\textbf{Question}]{\begin{trivlist}
\item[\hskip \labelsep {\bfseries #1}\hskip \labelsep {\bfseries #2.}]}{\end{trivlist}}
% Solution
\newenvironment{solution}{\begin{proof}[\textbf{Solution}]}{\end{proof}}

% Afterword
\newenvironment{afterword}[2][Appendix]{\begin{trivlist}
\item[\hskip \labelsep {\bfseries #1}\hskip \labelsep {\bfseries #2}]}{\end{trivlist}}

% ***********************************************************
% ********************** END NEW ENVS ***********************
% ***********************************************************

% ***********************************************************
% ********************** END TITLEPAGE **********************
% ***********************************************************