% Modified Template by Jonathan Doucette and Kevin Multani, original by: Jonathan Ward

% Modified Template by Jonathan Doucette and Kevin Multani, original by: Jonathan Ward

\documentclass[12pt]{article} 
\usepackage[english]{babel}
\usepackage[utf8]{inputenc}
\usepackage{amsmath} % AMS Math Package
\usepackage{amsthm} % Theorem Formatting
\usepackage{amssymb} % Math symbols such as \mathbb
\usepackage{graphicx} % Allows for eps images
\usepackage{multicol} % Allows for multiple columns
\usepackage[dvips,letterpaper,margin=1in,bottom=1in]{geometry}
\usepackage{hyperref}
\usepackage{parskip} % Removes indentation from paragraphs
\usepackage{xcolor,xspace,soul} % Colour, spacing, and highlighting
\usepackage{mathrsfs}
\usepackage{bm} % For bold math symbols
\usepackage{amscd}
\usepackage[all,cmtip]{xy}
%\usepackage{bbm}
\usepackage{titling}
\usepackage{listing} % for code snippets
%\usepackage{minted} % for code snippets
\usepackage{enumerate}
\usepackage{fancyhdr}
\usepackage[]{physics}
\usepackage[makeroom]{cancel}
\usepackage{pdfpages}
\usepackage[]{mcode}
\usepackage[title]{appendix}

% ***********************************************************
% ********************** BEGIN TITLE PAGE *******************
% ***********************************************************
\newcommand*{\titleGM}{\begingroup % Create the command for including the title page in the document
\hbox{ % Horizontal box
\hspace*{0.2\textwidth} % Whitespace to the left of the title page
\rule{1pt}{\textheight} % Vertical line
\hspace*{0.05\textwidth} % Whitespace between the vertical line and title page text
\parbox[b]{0.75\textwidth}{ % Paragraph box which restricts text to less than the width of the page

{\noindent\Huge\bfseries MATH 521}\\[2\baselineskip] % Title
{\large \textit{Assignment 4}}\\[4\baselineskip]
%{\large \textsc{ Jonathan Doucette }} % Author name

\vspace{0.5\textheight} % Whitespace between the title block and the publisher
{\noindent \today }\\[\baselineskip]
%{\noindent Student Number: 35298124 }\\[\baselineskip] 
%{\noindent \footnotesize All problems below are \textit{Copyright \copyright \space 2018 Timm Treskatis. All Rights Reserved.} }\\[\baselineskip]
} % end parbox
} % end hbox
\endgroup}

 % Sets margins and page size
\pagestyle{fancy} 

%\lhead{Jonathan Doucette}
\rhead{\today}
\rfoot{Page \thepage}
\cfoot{}

\makeatletter % Need for anything that contains an @ command 
\renewcommand{\maketitle} % Redefine maketitle to conserve space
{ \begingroup \vskip 10pt \begin{center} \Huge {\bf \@title}
\vskip 10pt \large \@author \hskip 20pt \@date \end{center}
  \vskip 10pt \endgroup \setcounter{footnote}{0} }
\makeatother % End of region containing @ commands

% ***********************************************************
% ********************** END TITLE PAGE *********************
% ***********************************************************

% ***********************************************************
% ********************** BEGIN NEW COMMANDS *****************
% ***********************************************************

\renewcommand{\labelenumi}{(\alph{enumi})} % Use letters for enumerate
\let\vaccent=\v % rename builtin command \v{} to \vaccent{}

%% MISC
\newcommand{\ab}[1]{\left| #1 \right|} % for absolute value
\newcommand{\avg}[1]{\left< #1 \right>} % for average
\let\underdot=\d % rename builtin command \d{} to \underdot{}
\let\baraccent=\= % rename builtin command \= to \baraccent
\renewcommand{\=}[1]{\stackrel{#1}{=}} % for putting numbers above =
\providecommand{\fr}{\frac}
\providecommand{\RR}{\mathbb{R}}
\providecommand{\CC}{\mathbb{C}}
\providecommand{\NN}{\mathbb{N}}
\providecommand{\e}{\epsilon}
\DeclareMathOperator{\di}{d\!}
\newcommand*\ieval[3]{\left.#1\right\rvert_{#2}^{#3}}

%% Vectors
\renewcommand{\v}[1]{\ensuremath{\mathbf{#1}}} 
\newcommand{\gv}[1]{\ensuremath{\mbox{\boldmath$ #1 $}}} % for vectors of Greek letters
\newcommand{\uv}[1]{\ensuremath{\mathbf{\hat{#1}}}} % for unit vector
\providecommand{\wave}[1]{\v{\tilde{#1}}}

%% DERIVATIVES
\renewcommand{\d}[2]{\frac{d #1}{d #2}} % for derivatives
\newcommand{\dubd}[2]{\frac{d^2 #1}{d #2^2}} % for double derivatives
\newcommand{\pd}[2]{\frac{\partial #1}{\partial #2}} % for partial derivatives
\newcommand{\pdd}[2]{\frac{\partial^2 #1}{\partial #2^2}} % for double partial derivatives

%% Operators
\newcommand{\Gradient}{\ensuremath{\mbox{\boldmath$\nabla$}}} % gradient

%% Text
\newcommand{\mathcolorbox}[2]{\colorbox{#1}{$\displaystyle #2$}}
\newcommand{\hlfancy}[3]{\textcolor{#1}{\sethlcolor{#2}\hl{#3}}}
\newcommand{\TODO}[1]{\hlfancy{red}{yellow}{\textbf{TODO: #1}}}

%% Code
%\newcommand{\code}[1]{\mintinline{C}{#1}}
%\newcommand{\code}[1]{\texttt{#1}}
\newcommand{\code}[1]{\lstinline[columns=fixed]{#1}}
\newcommand{\includecode}[1]{\lstinputlisting{#1}}

% ***********************************************************
% ********************** END NEW COMMANDS *******************
% ***********************************************************

% ***********************************************************
% ********************** BEGIN NEW ENVS *********************
% ***********************************************************

% Theorem
\newenvironment{theorem}[2][Theorem]{\begin{trivlist}
\item[\hskip \labelsep {\bfseries #1}\hskip \labelsep {\bfseries #2.}]}{\end{trivlist}}
% Lemma
\newenvironment{lemma}[2][Lemma]{\begin{trivlist}
\item[\hskip \labelsep {\bfseries #1}\hskip \labelsep {\bfseries #2.}]}{\end{trivlist}}
% Corollary
\newenvironment{corollary}[2][Corollary]{\begin{trivlist}
\item[\hskip \labelsep {\bfseries #1}\hskip \labelsep {\bfseries #2.}]}{\end{trivlist}}

% Exercise
\newenvironment{exercise}[2][Exercise]{\begin{trivlist}
\item[\hskip \labelsep {\bfseries #1}\hskip \labelsep {\bfseries #2.}]}{\end{trivlist}}
% Problem
\newenvironment{problem}[2][Problem]{\begin{trivlist}
\item[\hskip \labelsep {\bfseries #1}\hskip \labelsep {\bfseries #2.}]}{\end{trivlist}}
% Question
\newenvironment{question}[2][Question]{\begin{trivlist}
\item[\hskip \labelsep {\bfseries #1}\hskip \labelsep {\bfseries #2.}]}{\end{trivlist}}
% Solution
\newenvironment{solution}{\begin{proof}[Solution]}{\end{proof}}

% Afterword
\newenvironment{afterword}[2][Appendix]{\begin{trivlist}
\item[\hskip \labelsep {\bfseries #1}\hskip \labelsep {\bfseries #2}]}{\end{trivlist}}

% ***********************************************************
% ********************** END NEW ENVS ***********************
% ***********************************************************

% ***********************************************************
% ********************** END TITLEPAGE **********************
% ***********************************************************

\begin{document}

\begin{titlingpage}
	\titleGM
\end{titlingpage}
\clearpage
\setcounter{page}{1}

%%%%%%%%%%%%%%%%%%%%%%%%%
% ----- PROBLEM 1 ----- %
%%%%%%%%%%%%%%%%%%%%%%%%%
\begin{problem}{1}
Consider the steady advection-diffusion equation in 1D
%
\begin{align*}
    a u' - D u'' &= 0 \qquad
    \text{in } ]0,1[\\
    u(0) &= 0 \\
    u(1) &= 1
\end{align*}
%
where the advection velocity $a \ge 0$ and the diffusivity 
$D>0$ are constant throughout the domain.

% ----- Problem 1(a) ----- %
\begin{itemize}
	\item[(a)] Discretise the advection-diffusion problem
	using the $N+1$ grid points
	$$x=0,\:h,\:2h,\:3h,\:\ldots,\:(N-1)h,\:1$$
	(where $h=1/N$) with the second-order consistent central 
	difference approximation
	$$u^\prime(x)\approx\frac{u(x+h)-u(x-h)}{2h}.$$
	Write down the 'big linear system' $L^h_{CD}u^h=f^h$
	(the first two rows and the last row suffice).
	
	For what range of $h$-values does the matrix $L^h_{CD}$
	satisfy the $M$-matrix criterion 'weakly chained
	diagonally dominant \& $L$-matrix'?
	
	\textit{Hint:} The identity
	$$\ab{\alpha + \beta} + \ab{\alpha - \beta} =
	2\max\{\ab{\alpha},\ab{\beta}\}$$
	may be useful.         
\end{itemize}

% ----- Problem 1(a): Solution ----- %
\begin{solution}
The matrix $L^h_{CD}$ in the 'big linear system' $L^h_{CD}u^h=f^h_{CD}$
may be written as the sum of the advection matrix $A^h_{CD}$ and the
(negative-)diffusion matrix $D^h$, where
\begin{align*}
    A^h_{CD} = \frac{a}{2h}
    \begin{bmatrix}
		0  &  1 &         &    \\
	   -1  &  0 &  1      &    \\
		   &    &  \ddots &    \\
		   &    &      -1 &  0
    \end{bmatrix}
    \qquad\text{and}\qquad
    D^h = \frac{D}{h^2}
    \begin{bmatrix}
	    2  & -1  &         &    \\
	   -1  &  2  & -1      &    \\
		   &     &  \ddots &    \\
		   &     &      -1 &  2
	\end{bmatrix}.
\end{align*}
%
The vector $f^h_{CD}$ is given by all zeros except for the last entry,
\begin{align*}
	f^h_{CD} =
    \begin{bmatrix}
	    0, & 0, & \hdots, & 0, & -\frac{a}{2h} + \frac{D}{h^2}
	\end{bmatrix}^T,
\end{align*}
and so the total linear system $L^h_{CD} u^h = f^h_{CD}$ is given by
\begin{align*}
    \begin{bmatrix}
	    \frac{2D}{h^2} & \frac{a}{2h} - \frac{D}{h^2} &  & \\
	   -\frac{a}{2h} - \frac{D}{h^2}  & \frac{2D}{h^2} & \frac{a}{2h} -\frac{D}{h^2} & \\
	   & & \ddots & \\
	   & & -\frac{a}{2h} - \frac{D}{h^2} & \frac{2D}{h^2}
    \end{bmatrix}
    \begin{bmatrix}
        u_0 \\ u_1 \\ \vdots \\ u_{N-1}
    \end{bmatrix}
    =
    \begin{bmatrix}
	    0 \\ 0 \\ \vdots \\ -\frac{a}{2h} + \frac{D}{h^2}
	\end{bmatrix}.
\end{align*}

\pagebreak

In order to check the requirements on $h$ for $L^h_{CD}$ to satisfy
the $M$-matrix criteria, we first consider the range of $h$-values such
that that $L^h_{CD}$ is weakly chained diagonally dominant.
In order for this to hold, we 
require that $L^h_{CD}$ has the following three properties:
\begin{enumerate}[(i)]
    \item $L^h_{CD}$ is weakly diagonally dominant in all rows
    \item $L^h_{CD}$ is strongly diagonally dominant in at least one row
    \item For all rows $i_0$ there exists a chain of indices $i_0
    \rightarrow i_1 \rightarrow \cdots \rightarrow i_s$ to a strictly
    diagonally dominant row $i_s$ such that all $a_{i_{l-1},i_l} \neq 0$
    $(l = 1, ..., s)$
\end{enumerate}

We will check property (i) first. Let $\alpha = a/2h \geq 0$ and 
$\beta = D/h^2 > 0$. For the intermediate rows, property (i) requires that
\begin{align}\label{Lcd_MidRows}
    2 \beta &\geq \ab{\alpha + \beta} + \ab{\alpha - \beta} \\
            &= 2\max\{\ab{\alpha},\ab{\beta}\} \nonumber \\
            &= 2\max\{\alpha,\beta\} \nonumber
\end{align}
Since $\beta > 0$ and $\alpha \geq 0$, it must be that 
$\beta \geq \alpha$, or in terms of the original variables,
\begin{align}\label{Lcd_Cond1}
    \frac{D}{h^2} \geq \frac{a}{2h}.
\end{align}
In other words, \textit{diffusive effects must be at least as
strong as advection effects}.
Now, the conditions on the first and last rows respectively are
\begin{gather}
    2\beta \geq \ab{\alpha - \beta}\label{Lcd_FirstRow} \\
    2\beta \geq \ab{-\alpha - \beta} = \ab{\alpha + \beta} = \alpha + \beta\label{Lcd_LastRow}
\end{gather}
The condition on the last row clearly follows from the fact that
$\beta \geq \alpha $ (and that both $\alpha$ and $\beta$ are 
non-negative). The condition on the first row follows afterwards
from the fact that
$$\ab{\alpha - \beta} \leq \ab{\alpha + \beta}
= \alpha + \beta \leq 2\beta.$$
Thus, condition \ref{Lcd_Cond1} is sufficient to satisfy (i).

For condition (ii), we first note that the left-hand side of conditions
\ref{Lcd_MidRows}, \ref{Lcd_FirstRow}, and \ref{Lcd_LastRow} are all
the same, namely $2\beta$. It is easy to see then that since
$$\abs{\alpha - \beta} \leq \alpha + \beta \leq 2\max\{\alpha,\beta\},$$
the weakest requirement on $L^h_{CD}$ would be to require the first
row to be strictly diagonally dominant, i.e. require
$2\beta > \ab{\alpha - \beta}$. Since we additionally have that
$\beta \geq \alpha \geq 0$, this is equivalent to 
$2\beta > \beta - \alpha$, or
$\beta > -\alpha$. Since $\alpha \geq 0$, this is trivially satisfied
by the fact that $\beta > 0$. Thus, the first row of the matrix is
strictly diagonally dominant, following condition \ref{Lcd_Cond1}.

Condition (iii) is satisfied by the tridiagonal structure of $L^h_{CD}$ 
if we further require that the first
super-diagonal of each row is non zero. This is equivalent to 
requiring that, in combination with condition \ref{Lcd_Cond1},
$\beta > \alpha$.
In terms of the original variables, we have that
\begin{align}\label{Lcd_Cond2}
    \boxed{\frac{D}{h^2} > \frac{a}{2h}}.
\end{align}

Similarly, $L^h_{CD}$ is clearly an $L$-matrix (positive diagonal
entries and non-positive off-diagonal entries) by the fact that
$\alpha < \beta$, $\beta > 0$, and $\alpha \geq 0$. Thus, condition
\ref{Lcd_Cond2} is sufficient to make $L^h_{CD}$ satisfy the $M$-matrix
criterion. Writing condition \ref{Lcd_Cond2} as a condition on $h$,
we require that
\begin{align}\label{Lcd_h_Cond}
    \boxed{ h < \frac{2D}{a}}.
\end{align}

\end{solution}
\pagebreak

% ----- Problem 1(b) ----- %
\begin{itemize}
	\item[(b)] Leaving everything else unchanged, discretise the 
	transport term with the first-order consistent upwind differencing
	scheme
	$$u'(x) \approx \frac{u(x)-u(x-h)}{h}$$
	instead. Write down the 'big linear system' $L^h_{UD}u^h=f^h$.
	
	For what range of $h$-values does the matrix $L^h_{UD}$
	satisfy the $M$-matrix criterion?
\end{itemize}

% ----- Problem 1(b): Solution ----- %
\begin{solution}
The advection matrix $A^h_{CD}$ has now been modified to the upwind
advection matrix $A^h_{UD}$, given by
\begin{align*}
    A^h_{UD} = \frac{a}{h}
    \begin{bmatrix}
		1  &  0 &         &    \\
	   -1  &  1 &  0      &    \\
		   &    &  \ddots &    \\
		   &    &      -1 &  1
    \end{bmatrix}.
\end{align*}
The (negative-)diffusion matrix $D^h$ remains as before, but the
right-hand side vector $f^h_{UD}$ no longer depends on the advection
process at all, and is given by
\begin{align*}
	f^h_{UD} =
    \begin{bmatrix}
	    0, & 0, & \hdots, & 0, & \frac{D}{h^2}
	\end{bmatrix}^T.
\end{align*}
Intuitively, this is because the advection process 'transfers 
information from left to right', and so the right endpoint is only
affected by it's neighbour through the diffusive process, as diffusion
transfers information isotropically in both directions.

The new 'big linear system' $L^h_{UD}u^h=f^h_{UD}$ is now given by
\begin{align*}
    \begin{bmatrix}
	    \frac{2D}{h^2} + \frac{a}{h} & - \frac{D}{h^2} &  & \\
	   -\frac{a}{h} - \frac{D}{h^2}  & \frac{2D}{h^2} + \frac{a}{h} & -\frac{D}{h^2} & \\
	   & & \ddots & \\
	   & & -\frac{a}{h} - \frac{D}{h^2} & \frac{2D}{h^2} + \frac{a}{h}
    \end{bmatrix}
    \begin{bmatrix}
        u_0 \\ u_1 \\ \vdots \\ u_{N-1}
    \end{bmatrix}
    =
    \begin{bmatrix}
	    0 \\ 0 \\ \vdots \\ \frac{D}{h^2}
	\end{bmatrix}.
\end{align*}

Now, it is trivial to see that $L^h_{UD}$ is a $L$-matrix, as since
$D>0$ and $a\geq 0$ we see that the diagonal is strictly positive
and the off diagonals are strictly negative.

Using the same notation as in problem 1(a), the condition on the middle
rows to be weakly diagonally dominant is
\begin{align}\label{Lud_MidRows}
    2\beta + \alpha
    \geq \ab{-\alpha -\beta} + \ab{-\beta}
    = 2\beta + \alpha,
\end{align}
which is trivially satisfied.

For the first and last row respectively, we have
\begin{gather}
    2\beta + \alpha \geq \ab{-\beta} = \beta\label{Lud_FirstRow} \\
    2\beta + \alpha \geq \ab{-\alpha -\beta} = \alpha + \beta,\label{Lud_FirstRow}
\end{gather}
both of which are trivially satisfied by the non-negativity of 
$\alpha$ and $\beta$. In fact, as $\beta > 0$, both the first and
last rows are \textit{strictly} diagonally dominant.

Therefore, we have that
\begin{enumerate}[(i)]
    \item $L^h_{UD}$ is weakly diagonally dominant in all rows
    \item $L^h_{UD}$ is strongly diagonally dominant in at least one row (both the first and the last)
    \item $L^h_{UD}$ is weakly chained, following the tridiagonal structure, the first row being strictly diagonally dominant, and all off-diagonal rows being non-zero (in fact, strictly negative)
\end{enumerate}
Therefore $L^h_{UD}$ is weakly chained diagonally dominant, and since
$L^h_{UD}$ is also an $L$-matrix, it follows that $L^h_{UD}$ is an
$M$-matrix.

\boxed{\textbf{There is no restriction on the grid spacing \bm{$h$}}},
as we have only assumed that $D>0$ and $a\geq 0$.

\end{solution}
\pagebreak


% ----- Problem 1(c) ----- %
\begin{itemize}
	\item[(c)] Download the file \code{advection_diffusion.m} and
	read the output of \code{help} \linebreak 
	\code{advection_diffusion}.
	Everything else is intentionally obfuscated so that
	you still have to do
	(a) and (b) yourself! You can however use this function to
	check your answers for the admissible ranges of $h$ in (a)
	and (b).
	
	What do you observe if the matrix $L^h$ is not an $M$-matrix?
	Use your knowledge from lectures and the technical language
	you have learnt so far to \textit{describe} and 
	\textit{explain} your observations.
\end{itemize}

% ----- Problem 1(c): Solution ----- %
\begin{solution}
The expectation in the investigation of the behaviour of the solution
is that if $L^h$ is an $M$-matrix, we have the monotonicity property
(where $\succeq$ denotes entry-wise $\geq$):
$$L^h u^h \succeq 0 \Rightarrow u^h \succeq 0. $$

We have shown that the matrix $L^h_{UD}$ is an $M$-matrix for all $h$,
and the matrix $L^h_{CD}$ is an $M$-matrix only for $h<2D/a$. Now,
in considering the right-hand side vector $f^h_{CD}$ and $f^h_{UD}$,
we note additionally that
\begin{gather*}
    f^h_{CD} \succeq 0 \Leftrightarrow h < \frac{2D}{a} \qquad
    \text{and} \qquad
    f^h_{UD} \succeq 0 \quad \forall h > 0.
\end{gather*}
Therefore, the solutions computed using the upwind differentiation scheme
should be strictly positive for any grid size $h$. For the centred scheme,
however, there is no such guarantee when $h \geq 2D/a$.
This is due to neither the matrix $L^h_{CD}$ being an $M$-matrix in this
case, nor the right-hand side vector $f^h_{CD}$ being non-negative.

Thus, although we expect the $L^h_{CD}$ system to converge to the true
solution faster as $h\rightarrow 0$ (second order in $h$ vs. first order),
we have no guarantees that the solution will behave well for $h \geq 2D/a$.
The solution will likely be erratic and non-physical.

Lastly, we expect the elliptic operator $L^h$ to obey the elliptic
maximum/minimum principle when $L^h$ is an $M$-matrix, but this is
not guaranteed when $L^h$ is not an $M$-matrix.

The script in Appendix \ref{q1c} was used to investigate the behaviour
of the respective solutions for varying values of $h$ both above and
below the critical value $2D/a$. As expected, the main results were as follows:
\begin{itemize}
    \item The solution to the upwind system $L^h_{UD} u^h = f^h_{UD}$
    was strictly positive for all grid sizes
    \item The solution to the centred system $L_h{CD} u^h = f^h_{CD}$
    was positive only for $h<2D/a$, and was non-negative for $h\leq 2D/a$
    \item The centered system solution was additionally highly erratic
    for $h>2D/a$, while the upwind system was smooth for all $h$
    \item As $h\rightarrow 0$, the centred system converged faster than
    the upwind system
    \item Both system obey the elliptic maximum/minimum principles when
    the $L^h$ matrices satisfy the $M$-criterion, but as solutions to
    to the centred system produce negative values for $h<2D/a$, the
    elliptic minimum principle is clearly not satisfied in this case.
\end{itemize}

\end{solution}
\pagebreak


% ----- Problem 1(d) ----- %
\begin{itemize}
	\item[(d)] Even though the upwind differencing scheme is only
	first-order consistent as an approximation of the equation
	$$ a u' - D u'' = 0 $$
	it is second-order consistent as an approximation of a slightly
	different equation,
	$$ a u' - (D + \tilde{D}) u'' = 0, $$
	provided that $u \in C^4([0,1])$. Calculate this number
	$\tilde{D}$.
	
	\textit{Hint:} Use Taylor expansions as done in class or
	video \#7 to determine the truncation error
	$$ a u'(x) \quad - \quad
	(\text{upwind difference approximation of this term}).$$
	
	$\tilde{D}$ will depend on h.
\end{itemize}

% ----- Problem 1(d): Solution ----- %
\begin{solution}
Given $u \in C^4([0,1])$, we can expand $u(x\pm h)$ to third and
fourth order taylor series in $h$ as
\begin{gather}
    u(x\pm h) = u(x) \pm h u'(x) + \frac{h^2}{2} u''(x) \pm 
    \frac{h^3}{6} u'''(\zeta_\pm)\label{u_3rd} \\
    u(x\pm h) = u(x) \pm h u'(x) + \frac{h^2}{2} u''(x) \pm 
    \frac{h^3}{6} u'''(x) + \frac{h^4}{24}u''''(\xi_\pm)\label{u_4th}
\end{gather}
where $\xi_+$ and $\zeta_+$ are in $[x,x+h]$, and 
$\xi_-$ and $\zeta_-$ are in $[x-h,x]$.

Now, we consider the error between $au'-(D+\tilde{D})u''$
and the upwind differencing scheme for $au' - Du''$.
Namely, we have that
\begin{gather*}
    \ab{ au'(x)-(D+\tilde{D})u''(x) -
    \bigg( a\frac{u(x)-u(x-h)}{h} -
    D\frac{u(x-h) - 2u(x) + u(x+h)}{h^2} \bigg) } \\
    = \ab{ a\bigg(u'(x) - \frac{u(x)-u(x-h)}{h}\bigg) -
    D\bigg( u''(x) - \frac{u(x-h) - 2u(x) + u(x+h)}{h^2} \bigg) -
    \tilde{D}u''(x) } \\
    = \ab{ a \bigg( \frac{h}{2}u''(x) - \frac{h^2}{6}u'''(\zeta_-) 
    \bigg) +
    D \bigg( \frac{h^2}{24} (u''''(\xi_+) + u''''(\xi_-)) \bigg) -
    \tilde{D} u''(x) } \\
    = \ab{ \bigg( \frac{ah}{2} - \tilde{D} \bigg)u''(x) +
    h^2 \bigg( \frac{D}{24}(u''''(\xi_+) + u''''(\xi_-)) -
    \frac{a}{6}u'''(\zeta_-) \bigg) } \\
    \leq h^2 \max_{[x-h,x+h]} \frac{a}{6}\ab{u'''} + 
    \frac{D}{12}\ab{u''''}
\end{gather*}
where in the last line we have chosen $\tilde{D}=ah/2$ and used
the triangle inequality to obtain an upper bound on the error in terms
of the third and fourth derivatives on the whole interval.

Thus, the upwind differencing scheme is second-order consistent
for the equation $$au' - (D+\tilde{D})u''=0$$ if we choose
$\boxed{\tilde{D} = \frac{ah}{2}}$.

\end{solution}
\pagebreak

\end{problem} % END PROBLEM 1
\pagebreak

%%%%%%%%%%%%%%%%%%%%%%%
% ----- WRITTEN ----- %
%%%%%%%%%%%%%%%%%%%%%%%
\begin{afterword}[Your Primer Talk]{}
Primer talk information:
\begin{itemize}
    \item \textbf{\underline{I do not plan on using the 
    document camera}}, I plan on preparing a short slideshow
    presentation.
    \item By the end of this talk, the audience should understand
    what the Bloch-Torrey equation is and how it's solution is both
    useful to the MRI community in terms of modelling MR signals,
    and that solving it in realistic geometries is difficult due to
    the multiple scales of the problem.
\end{itemize}
\end{afterword}
\vspace{1cm}

\begin{afterword}[Your Learning Progress]{}
\textit{What is the one most important thing that you have learnt in this
assignment?}

Question 1c was definitely the most illuminating part of this assignment.
I knew intuitively that something like this must have been true, but
I did not know before this the precise relation between stable 
solutions and the $M$-matrix criterion of the linear system and how
vital it is to having well behaved solutions.

\vspace{0.5cm}

\textit{What is the most substantial new insight that you have gained 
from this course this week? Any aha moment?}

The fact that first-order differencing schemes can be preferred over
second order schemes on physical and mathematical grounds was definitely
an \textit{aha moment}! I have heard about upwind differencing being
extremely important for stability in things like fluid equations,
but I never knew precisely why until now!
\end{afterword}

\pagebreak

%%%%%%%%%%%%%%%%%%%%
% ----- CODE ----- %
%%%%%%%%%%%%%%%%%%%%
\begin{appendices}
%\addcontentsline{toc}{section}{Appendix}
%\part{Appendix}
%\parttoc

\section{}\label{q1c}
\includecode{q1c.m}
\pagebreak
\section{}\label{advdiff}
\includecode{advection_diffusion.m}

\end{appendices}

\end{document}
