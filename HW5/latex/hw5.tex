% Modified Template by Jonathan Doucette and Kevin Multani, original by: Jonathan Ward

% Modified Template by Jonathan Doucette and Kevin Multani, original by: Jonathan Ward

\documentclass[12pt]{article} 
\usepackage[english]{babel}
\usepackage[utf8]{inputenc}
\usepackage{amsmath} % AMS Math Package
\usepackage{amsthm} % Theorem Formatting
\usepackage{amssymb} % Math symbols such as \mathbb
\usepackage{graphicx} % Allows for eps images
\usepackage{multicol} % Allows for multiple columns
\usepackage[dvips,letterpaper,margin=1in,bottom=1in]{geometry}
\usepackage{hyperref}
\usepackage{parskip} % Removes indentation from paragraphs
\usepackage{xcolor,xspace,soul} % Colour, spacing, and highlighting
\usepackage{mathrsfs}
\usepackage{bm} % For bold math symbols
\usepackage{amscd}
\usepackage[all,cmtip]{xy}
%\usepackage{bbm}
\usepackage{titling}
\usepackage{listing} % for code snippets
%\usepackage{minted} % for code snippets
\usepackage{enumerate}
\usepackage{fancyhdr}
\usepackage[]{physics}
\usepackage[makeroom]{cancel}
\usepackage{pdfpages}
\usepackage[]{mcode}
\usepackage[title]{appendix}

% ***********************************************************
% ********************** BEGIN TITLE PAGE *******************
% ***********************************************************
\newcommand*{\titleGM}{\begingroup % Create the command for including the title page in the document
\hbox{ % Horizontal box
\hspace*{0.2\textwidth} % Whitespace to the left of the title page
\rule{1pt}{\textheight} % Vertical line
\hspace*{0.05\textwidth} % Whitespace between the vertical line and title page text
\parbox[b]{0.75\textwidth}{ % Paragraph box which restricts text to less than the width of the page

{\noindent\Huge\bfseries MATH 521}\\[2\baselineskip] % Title
{\large \textit{Assignment 4}}\\[4\baselineskip]
%{\large \textsc{ Jonathan Doucette }} % Author name

\vspace{0.5\textheight} % Whitespace between the title block and the publisher
{\noindent \today }\\[\baselineskip]
%{\noindent Student Number: 35298124 }\\[\baselineskip] 
%{\noindent \footnotesize All problems below are \textit{Copyright \copyright \space 2018 Timm Treskatis. All Rights Reserved.} }\\[\baselineskip]
} % end parbox
} % end hbox
\endgroup}

 % Sets margins and page size
\pagestyle{fancy} 

%\lhead{Jonathan Doucette}
\rhead{\today}
\rfoot{Page \thepage}
\cfoot{}

\makeatletter % Need for anything that contains an @ command 
\renewcommand{\maketitle} % Redefine maketitle to conserve space
{ \begingroup \vskip 10pt \begin{center} \Huge {\bf \@title}
\vskip 10pt \large \@author \hskip 20pt \@date \end{center}
  \vskip 10pt \endgroup \setcounter{footnote}{0} }
\makeatother % End of region containing @ commands

% ***********************************************************
% ********************** END TITLE PAGE *********************
% ***********************************************************

% ***********************************************************
% ********************** BEGIN NEW COMMANDS *****************
% ***********************************************************

\renewcommand{\labelenumi}{(\alph{enumi})} % Use letters for enumerate
\let\vaccent=\v % rename builtin command \v{} to \vaccent{}

%% MISC
\newcommand{\ab}[1]{\left| #1 \right|} % for absolute value
\newcommand{\avg}[1]{\left< #1 \right>} % for average
\let\underdot=\d % rename builtin command \d{} to \underdot{}
\let\baraccent=\= % rename builtin command \= to \baraccent
\renewcommand{\=}[1]{\stackrel{#1}{=}} % for putting numbers above =
\providecommand{\fr}{\frac}
\providecommand{\RR}{\mathbb{R}}
\providecommand{\CC}{\mathbb{C}}
\providecommand{\NN}{\mathbb{N}}
\providecommand{\e}{\epsilon}
\DeclareMathOperator{\di}{d\!}
\newcommand*\ieval[3]{\left.#1\right\rvert_{#2}^{#3}}

%% Vectors
\renewcommand{\v}[1]{\ensuremath{\mathbf{#1}}} 
\newcommand{\gv}[1]{\ensuremath{\mbox{\boldmath$ #1 $}}} % for vectors of Greek letters
\newcommand{\uv}[1]{\ensuremath{\mathbf{\hat{#1}}}} % for unit vector
\providecommand{\wave}[1]{\v{\tilde{#1}}}

%% DERIVATIVES
\renewcommand{\d}[2]{\frac{d #1}{d #2}} % for derivatives
\newcommand{\dubd}[2]{\frac{d^2 #1}{d #2^2}} % for double derivatives
\newcommand{\pd}[2]{\frac{\partial #1}{\partial #2}} % for partial derivatives
\newcommand{\pdd}[2]{\frac{\partial^2 #1}{\partial #2^2}} % for double partial derivatives

%% Operators
\newcommand{\Gradient}{\ensuremath{\mbox{\boldmath$\nabla$}}} % gradient

%% Text
\newcommand{\mathcolorbox}[2]{\colorbox{#1}{$\displaystyle #2$}}
\newcommand{\hlfancy}[3]{\textcolor{#1}{\sethlcolor{#2}\hl{#3}}}
\newcommand{\TODO}[1]{\hlfancy{red}{yellow}{\textbf{TODO: #1}}}

%% Code
%\newcommand{\code}[1]{\mintinline{C}{#1}}
%\newcommand{\code}[1]{\texttt{#1}}
\newcommand{\code}[1]{\lstinline[columns=fixed]{#1}}
\newcommand{\includecode}[1]{\lstinputlisting{#1}}

% ***********************************************************
% ********************** END NEW COMMANDS *******************
% ***********************************************************

% ***********************************************************
% ********************** BEGIN NEW ENVS *********************
% ***********************************************************

% Theorem
\newenvironment{theorem}[2][Theorem]{\begin{trivlist}
\item[\hskip \labelsep {\bfseries #1}\hskip \labelsep {\bfseries #2.}]}{\end{trivlist}}
% Lemma
\newenvironment{lemma}[2][Lemma]{\begin{trivlist}
\item[\hskip \labelsep {\bfseries #1}\hskip \labelsep {\bfseries #2.}]}{\end{trivlist}}
% Corollary
\newenvironment{corollary}[2][Corollary]{\begin{trivlist}
\item[\hskip \labelsep {\bfseries #1}\hskip \labelsep {\bfseries #2.}]}{\end{trivlist}}

% Exercise
\newenvironment{exercise}[2][Exercise]{\begin{trivlist}
\item[\hskip \labelsep {\bfseries #1}\hskip \labelsep {\bfseries #2.}]}{\end{trivlist}}
% Problem
\newenvironment{problem}[2][Problem]{\begin{trivlist}
\item[\hskip \labelsep {\bfseries #1}\hskip \labelsep {\bfseries #2.}]}{\end{trivlist}}
% Question
\newenvironment{question}[2][Question]{\begin{trivlist}
\item[\hskip \labelsep {\bfseries #1}\hskip \labelsep {\bfseries #2.}]}{\end{trivlist}}
% Solution
\newenvironment{solution}{\begin{proof}[Solution]}{\end{proof}}

% Afterword
\newenvironment{afterword}[2][Appendix]{\begin{trivlist}
\item[\hskip \labelsep {\bfseries #1}\hskip \labelsep {\bfseries #2}]}{\end{trivlist}}

% ***********************************************************
% ********************** END NEW ENVS ***********************
% ***********************************************************

% ***********************************************************
% ********************** END TITLEPAGE **********************
% ***********************************************************

\begin{document}

\begin{titlingpage}
	\titleGM
\end{titlingpage}
\clearpage
\setcounter{page}{1}

%%%%%%%%%%%%%%%%%%%%%%%%%
% ----- PROBLEM 1 ----- %
%%%%%%%%%%%%%%%%%%%%%%%%%
\begin{problem}{1}
In this assignment, we consider the linear elasticity problem
\begin{align}\label{LinElast_Strong}
\begin{split}
    -c \Laplacian u + au &= f \quad \text{in } \Omega \\
    u &= g \quad \text{on } \partial \Omega
\end{split}
\end{align}

on a polygonal domain $\Omega$. The function $u$ can be interpreted
as the elongation of a rubber membrane over the
$x_1x_2$-plane. The boundary values $g$ prescribe the elongation on 
$\partial\Omega$, e.g. by means of a wire frame construction
in which the membrane has been fixed. The real number $c > 0$
is the stiffness of the rubber material,
$a > 0$ is a constant proportional to its mass density and the
inhomogeneity $f$ models external forces that act on the membrane.

% ----- Problem 1(a) ----- %
\begin{itemize}
	\item[(a)]  Show that under the assumption of homogeneous boundary
	conditions, $g = 0$, the discretisation of \eqref{LinElast_Strong}
	with linear finite elements reads
	$$(cK^h + aM^h)\vec{u}^h = \vec{f}^h$$
	where
	\begin{align*}
	    k_{ij}^h &= \int_{\Omega} \nabla\phi_i^h \cdot
	    \nabla\phi_j^h \dx{x} \\
	    m_{ij}^h &= \int_{\Omega} \phi_i^h \phi_j^h \dx{x} \\
	    f_i^h &= \int_{\Omega} f \phi_i^h \dx{x} \\
	    \phi_i^h &= \text{hat function centred at the $i$-th vertex}
	\end{align*}
	for $i,j = 1,...,N$. $N$ is the number of effective
	degrees of freedom,
	i.e. the number of interior grid points
	which are not located on the boundary $\partial\Omega$.
	Note that since the domain is assumed to be a polygon,
	we can cover it exactly with a triangulation $T^h$
	such that $\Omega = \Omega^h$ (there is no mismatch on the boundary).    
\end{itemize}

% ----- Problem 1(a): Solution ----- %
\begin{solution}

We will transform the strong form \eqref{LinElast_Strong} of the problem
in to the weak form in several steps, as in class.

\begin{itemize}

\item We begin with the strong form \eqref{LinElast_Strong} of the problem.
Here, $u\in C^2(\Omega) \cap C(\bar{\Omega})$.

\item Next, as we are now looking for solutions to the weak form of the
problem, we multiply by a test function
$v \in H^1_0(\Omega)$ and
integrate over the domain to obtain
$$ -c \int_\Omega \Laplacian u v \dx{x} + a \int_\Omega u v \dx{x}
= \int_\Omega f v \dx{x} $$.

\item Now, we use Green's first identity
\begin{align*}
    \int_\Omega (\psi \Laplacian \varphi + 
    \nabla \psi \cdot \nabla \varphi) \dx{x} =
    \oint_{\partial\Omega} \psi (\nabla \varphi \cdot \hat{n}) \dx{x}
\end{align*}
to integrate the first term by parts. Using the fact that both $u$ and
$v$ are zero on the boundary, we arrive at the weak form of the problem:
\begin{align}\label{LinElast_Weak}
c \int_\Omega \nabla u \cdot \nabla v \dx{x} + 
a \int_\Omega u v \dx{x} = 
\int_\Omega f v \dx{x}.
\end{align}

\item We aim to find solutions $u \in H^1_0(\Omega)$ to the weak
problem \eqref{LinElast_Weak}. We project the true solution
$u$ onto the subspace $V^h \subset H^1_0(\Omega)$,
where $V^h$ is the space of functions spanned
by the linear ``hat'' basis functions $\phi_i^h(x)$,
where $i=1,...,N$.
Note that it is indeed a true subspace due to the assumption of the domain
$\Omega$ to be polygonal.
The hat function $\phi_i^h(x)$ takes the value $1$ on the $i$-th vertex,
decreases linearly to $0$ on all adjacent vertices,
and is $0$ on every other vertex.

We expand $u$ as a sum over the linear hat functions as
\begin{align*}
    u^h(x) = \sum_{j=1}^N \phi_j^h(x) u_j^h.
\end{align*}
The approximate solution $u^h$ represents a finite dimensional
approximation to the infinite dimensional solution $u$.

\item It is sufficient to assert that the weak form holds
against every linear basis function
$v_i=\phi_i^h \in V^h$ for $i = 1,...,N$.
This leads to the system of equations
\begin{align*}
    \sum_{i=1}^N \bigg(
    c \int_\Omega \nabla \phi_i^h \cdot \nabla \phi_j^h \dx{x} + 
    a \int_\Omega \phi_i^h \phi_j^h \dx{x} 
    \bigg) u_j^h = 
    \int_\Omega f \phi_i^h \dx{x}.
\end{align*}
The solution $u^h$ to this system of equations is optimal in the sense
of Galerkin, wherein the error vector $u^h-u$ is orthogonal to all
test functions $\phi_j(x)$.

\item Interpreting the left-hand side sum as a matrix multiplication
against a vector $\vec{u}^h = [u_1^h,...,u_N^h]^T$,
we now write
\begin{align*}
    k_{ij}^h &= \int_{\Omega} \nabla\phi_i^h \cdot
    \nabla\phi_j^h \dx{x} \\
    m_{ij}^h &= \int_{\Omega} \phi_i^h \phi_j^h \dx{x} \\
    f_i^h &= \int_{\Omega} f \phi_i^h \dx{x},
\end{align*}
where $k_{ij}^h$ and $m_{ij}^h$ are the $ij$-th elements of the matrices
$K^h$ and $M^h$, respectively, and $f_i^h$ is the $i$-th element
of the vector $\vec{f}^h$. The resulting system of equations
is the desired linear system
\begin{align}
    (cK^h + aM^h)\vec{u}^h = \vec{f}^h.
\end{align}

\end{itemize}

\end{solution}
\pagebreak

% ----- Problem 1(b) ----- %
\begin{itemize}
	\item[(b)] We can decompose the integrals that appear in the
	definition of the mass matrix into contributions from each
	triangle:
	\begin{align*}
	m_{ij}^h = \int_\Omega \phi_i^h \phi_j^h \dx{x} = 
	\sum_{k=1}^{n_T} \int_{T_k} \phi_i^h \phi_j^h \dx{x}
    \end{align*}
    $n_T$ is the number of triangles in the triangulation $T^h$.
    Let's look at one such triangle $T$, the vertices of which have the
    indices 1, 2 and 3. Note that only the three hat functions
    $\phi_1^h$, $\phi_2^h$, and $\phi_3^h$
    are non-zero on this triangle.
    
    Show that the element mass matrix with entries
    \begin{align*}
	m_{ij,T}^h = \int_T \phi_i^h \phi_j^h \dx{x} \qquad (i,j = 1,2,3)
	\end{align*}
	is
	\begin{align*}
	M_T^h = \frac{\ab{T}}{12}
	\begin{pmatrix}
	2 & 1 & 1 \\ 
	1 & 2 & 1 \\ 
	1 & 1 & 2
	\end{pmatrix}
	\end{align*}
	Hint: If you enjoy doing double integrals over triangles,
	you could use equation (2.21) from the notes. However,
	we don't want to mark endless calculations, so please use a
	different approach instead, similar to the mass
	matrix in 1D (Example 2.3.1, also available on Canvas under
	'Pages' in the menu).
	Your solution should fit on the remainder of this page.
	    
\end{itemize}

% ----- Problem 1(b): Solution ----- %
\begin{solution}
Products of linear basis functions are trivial to integrate on the
triangular domain if we use quadrature rules which are exact for
curves of degree $\leq 2$. One such rule is the
\textit{trapezoidal rule},
in which the integrand is evaluated at the midpoints of the edges
of the triangle $T$ and summed with 
equal weightings of $w_i = 1/3$ for $i=1,2,3$.

Since by definition the value of the linear basis functions
is linear from node to node, it is clear that the value at the
midpoint of the edge
is either $\frac{1}{2}$ i.e. the edge connects node values of
$0$ and $1$, or $0$ i.e. the edge
connects nodes each of value $0$.

Now, the integration is trivial. We consider two cases:
\begin{itemize}
\item[(i)] $i=j$, and so
$m_{ii,T}^h = \int_T (\phi_i^h)^2 \dx{x} = 
[ \frac{1}{3} (\frac{1}{2})^2 +
\frac{1}{3} (\frac{1}{2})^2 +
\frac{1}{3} (0)^2 ] \ab{T} = \frac{2}{12} \ab{T} $
\item[(ii)] $i\neq j$, and so
$m_{ij,T}^h = \int_T \phi_i^h \phi_j^h \dx{x} = 
[ \frac{1}{3} (\frac{1}{2})(\frac{1}{2}) +
\frac{1}{3} (\frac{1}{2})(0) +
\frac{1}{3} (0)(\frac{1}{2}) ] \ab{T} = \frac{1}{12} \ab{T} $
\end{itemize}

The resulting element mass matrix is therefore
\begin{align*}
M_T^h
= \begin{bmatrix}
m_{ij,T}^h
\end{bmatrix}
= \frac{\ab{T}}{12}
\begin{pmatrix}
2 & 1 & 1 \\ 
1 & 2 & 1 \\ 
1 & 1 & 2
\end{pmatrix}
\end{align*}
as desired.

\end{solution}
\pagebreak

\end{problem} % END PROBLEM 1
\pagebreak

%%%%%%%%%%%%%%%%%%%%%%%%%
% ----- PROBLEM 2 ----- %
%%%%%%%%%%%%%%%%%%%%%%%%%
\begin{problem}{2}
Download the file \code{discretiseLinearElasticity.m}.
We will turn this function into a
finite-element solver for Problem (1) next week.
Today we implement some core components.
The files \code{video10.mat} and \code{kiwi.mat}
contain arrays $P$, $E$ and $T$ which define a triangulation
on a polygonal computational domain $\Omega^h$.
Note that some versions of MATLAB's
plotting functions from the PDE Toolbox require extra
rows in $E$ and $T$.
If you are not using the PDE Toolbox, then you may delete all
but the first two rows of $E$ and all
but the first three rows of $T$,
as described in video \#10.
To import the variables from \code{video10.mat} or 
\code{kiwi.mat} into a structure \code{msh},
you may use the load command.

% ----- Problem 2(a) ----- %
\begin{itemize}
	\item[(a)] Unlike the triangle in Question 1(b),
	the actual vertices of the $k$-th triangle are probably
	not 1, 2 and 3. For instance, the 5th triangle in video \#10
	has the vertices 7, 10 and 9.
	In general, the $k$-th triangle has the vertices
	\code{T(1,k)}, \code{T(2,k)} and \code{T(3,k)}.
	
	Use Question 1(b) to complete the main function and the
	\code{assembleMass} subfunction.
	Can you do it without for loops?
	
	\textit{Hint:} In GNU Octave / MATLAB,
	the command sparse may be helpful.
\end{itemize}

% ----- Problem 2(a): Solution ----- %
\begin{solution} 
The \code{assembleMass} subfunction is as follows:
\includecode{assembleMass.m}

\end{solution}
\pagebreak

% ----- Problem 2(b) ----- %
\begin{itemize}
	\item[(b)] Write a script \code{hw5.m} to plot the triangular mesh
	and the sparsity pattern of the mass matrix that the
	function \code{discretiseLinearElasticity} returns
	(you don’t have to remove the rows/columns corresponding
	to boundary points). Do this for both data sets \code{video10.mat}
	and \code{kiwi.mat}.
	Make sure your plots are not distorted by using the
	\code{axis equal} command.
	
	\textit{Hint:} In installations of MATLAB with the PDE Toolbox,
	the command \code{pdemesh} may be helpful.
	In GNU Octave and MATLAB without the PDE Toolbox,
	the command \code{trimesh} may be helpful.
\end{itemize}

% ----- Problem 2(b): Solution ----- %
\begin{solution}
The \code{hw5.m} script is as follows:
\includecode{hw5_b.m}
\end{solution}
\pagebreak

% ----- Problem 2(c) ----- %
\begin{itemize}
	\item[(c)] Add extra commands to this script to plot your favourite
	function $u^h$ on the kiwi domain and compute its
	$L^2$-norm. Constant functions are not allowed!
	Make sure your plots are not distorted.
	
	\textit{Hint:} The commands \code{pdeplot} or \code{trisurf}
	may be helpful.
\end{itemize}

% ----- Problem 2(c): Solution ----- %
\begin{solution}
The modified \code{hw5.m} script is as follows:
\includecode{hw5_c.m}
\end{solution}
\pagebreak

\end{problem} % END PROBLEM 2
\pagebreak


%%%%%%%%%%%%%%%%%%%%%%%
% ----- WRITTEN ----- %
%%%%%%%%%%%%%%%%%%%%%%%
\begin{afterword}[Your Learning Progress]{}
What is the one most important thing that you have learnt
in this assignment?

\vspace{0.2cm}

I learned how to deal with the mess of indices that are finite
element meshes! The pros of FEM seems to outweigh the cons so
far, though - fumbling with indices is a small price to pay
for being able to solve PDEs on irregular domains.

\vspace{0.8cm}

What is the most substantial new insight that you have gained 
from this course this week? Any \textit{aha moment}?

\vspace{0.2cm}

I didn't have any particular \textit{aha moment}, although
finally getting the FEM indices correct (and without any
\code{for}-loops) was satisfying, especially considering that
all-in-all it looks rather elegant.

\end{afterword}

\pagebreak

%%%%%%%%%%%%%%%%%%%%
% ----- CODE ----- %
%%%%%%%%%%%%%%%%%%%%
\begin{appendices}
%\addcontentsline{toc}{section}{Appendix}
%\part{Appendix}
%\parttoc

%\section{}\label{q1c}
%\includecode{q1c.m}
%\pagebreak
%\section{}\label{advdiff}
%\includecode{advection_diffusion.m}

\end{appendices}

\end{document}
