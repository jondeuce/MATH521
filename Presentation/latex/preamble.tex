\documentclass[11pt]{beamer}
%\usetheme{Warsaw}
\usetheme{Madrid}
\usepackage[utf8]{inputenc}
\usepackage{amsmath}
\usepackage{amsfonts}
\usepackage{amssymb}
\usepackage{graphicx}
\author{Jonathan Doucette}
\title[Numerical methods in MRI]{Numerical methods in magnetic resonance imaging: the Bloch-Torrey equation}
\setbeamercovered{transparent} 
%\setbeamertemplate{navigation symbols}{}
%\logo{}
\institute[UBC]{University of British Columbia}
\date{\today}
\subject{MATH 521}

% ************************************************
% ***************** NEW COMMANDS *****************
% ************************************************

%\renewcommand{\labelenumi}{(\alph{enumi})} % Use letters for enumerate
%\let\vaccent=\v % rename builtin command \v{} to \vaccent{}

%% MISC
\newcommand{\ab}[1]{\left| #1 \right|} % for absolute value
\newcommand{\avg}[1]{\left< #1 \right>} % for average
\let\underdot=\d % rename builtin command \d{} to \underdot{}
\let\baraccent=\= % rename builtin command \= to \baraccent
\renewcommand{\=}[1]{\stackrel{#1}{=}} % for putting numbers above =
\providecommand{\fr}{\frac}
\providecommand{\RR}{\mathbb{R}}
\providecommand{\CC}{\mathbb{C}}
\providecommand{\NN}{\mathbb{N}}
\providecommand{\e}{\epsilon}
\DeclareMathOperator{\di}{d\!}
\newcommand*\ieval[3]{\left.#1\right\rvert_{#2}^{#3}}

%% Vectors
\renewcommand{\v}[1]{\ensuremath{\mathbf{#1}}} 
\newcommand{\gv}[1]{\ensuremath{\mbox{\boldmath$ #1 $}}} % for vectors of Greek letters
\newcommand{\uv}[1]{\ensuremath{\mathbf{\hat{#1}}}} % for unit vector
\providecommand{\wave}[1]{\v{\tilde{#1}}}

%% DERIVATIVES
\renewcommand{\d}[2]{\frac{d #1}{d #2}} % for derivatives
\newcommand{\dubd}[2]{\frac{d^2 #1}{d #2^2}} % for double derivatives
\newcommand{\pd}[2]{\frac{\partial #1}{\partial #2}} % for partial derivatives
\newcommand{\pdd}[2]{\frac{\partial^2 #1}{\partial #2^2}} % for double partial derivatives
\newcommand{\dx}[1]{\ensuremath{\operatorname{d}\!{#1}}}

%% Operators
\newcommand{\Gradient}{\ensuremath{\mbox{\boldmath$\nabla$}}} % gradient
\newcommand{\Laplacian}{\ensuremath{\mbox{$\Delta$}}} % laplacian

%% Norms
\newcommand{\Norm}[1]{\ensuremath{\lVert #1 \rVert}} % generic norm
\newcommand{\ANorm}[2]{\ensuremath{\lVert #1 \rVert_{#2}}} % norm with respect to specified A
\newcommand{\LTwoNorm}[1]{\ensuremath{\lVert #1 \rVert_{L^2}}} % L2 norm
\newcommand{\HOneNorm}[1]{\ensuremath{\lVert #1 \rVert_{H^1}}} % H1 norm

%% Math Operators
\DeclareMathOperator{\spann}{span}

% ***************************************************
% ******************* TEXT AND CODE *****************
% ***************************************************

%% Text
\newcommand{\mathcolorbox}[2]{\colorbox{#1}{$\displaystyle #2$}}
\newcommand{\hlfancy}[3]{\textcolor{#1}{\sethlcolor{#2}\hl{#3}}}
\newcommand{\TODO}[1]{\hlfancy{red}{yellow}{\textbf{TODO: #1}}}

%% Code
%\newcommand{\code}[1]{\mintinline{C}{#1}}
%\newcommand{\code}[1]{\texttt{#1}}
%\newcommand{\code}[1]{\lstinline[columns=fixed]{#1}}
%\newcommand{\includecode}[1]{\lstinputlisting{#1}}
