%% ------------------------------------------------------ %%
%% ------------ Additional Packages/Commands ------------ %%
%% ------------------------------------------------------ %%
\usepackage[T1]{fontenc}
\usepackage[utf8]{inputenc}
\usepackage{newunicodechar}
\usepackage[margin=0.6in,heightrounded]{geometry}
\usepackage{textcomp}
\usepackage{gensymb}
\usepackage{amssymb}
\usepackage{amsmath}
\usepackage{amsfonts}
\usepackage{mathtools}
\usepackage{mathrsfs}
\usepackage{eucal}
\usepackage{units}
\usepackage{siunitx}
\usepackage{color}
\usepackage{bm} % for bold vectors
\usepackage{upgreek} % for bold greek-letter vectors
%\usepackage{algorithmicx}
\usepackage{algorithm,algorithmic,setspace} % http://ctan.org/pkg/algorithms
%\usepackage[noend]{algpseudocode} % http://ctan.org/pkg/algorithmicx
\usepackage[section]{placeins}
\usepackage{afterpage} % for clearing page after figure
\usepackage{titling} % for title
\usepackage{caption} % for caption in minipage environment
\usepackage{authblk} % for affiliations
\usepackage{parskip} % Removes indentation from paragraphs
%\usepackage[bottom]{footmisc}
%\raggedbottom % allow paragraphs to end before page bottom
%\usepackage{graphicx}
%\usepackage{textpos} %for positioning logo
%\usepackage{animate} %for animated gif
\usepackage{xcolor,xspace,soul} % Colour, spacing, and highlighting
\usepackage{calc}

% Spacing
\setlength{\footskip}{\paperheight-(1.0in+\voffset+\topmargin+\headheight+\headsep+\textheight)-0.3in}
\AtBeginDocument{%
 \abovedisplayskip=8pt
 \abovedisplayshortskip=8pt
 \belowdisplayskip=8pt
 \belowdisplayshortskip=8pt
}

% ************************************************
% ***************** NEW COMMANDS *****************
% ************************************************

%\renewcommand{\labelenumi}{(\alph{enumi})} % Use letters for enumerate
\let\vaccent=\v % rename builtin command \v{} to \vaccent{}

%% MRI Specific
\newcommand{\conj}[1]{\overline{#1}}
%\newcommand{\Mxy}{\widetilde{M}_{xy}}
%\newcommand{\Mxy}{\widetilde{M}}
\newcommand{\Mxy}{\mathcal{M}}
\newcommand{\CDecay}{\Gamma}
\newcommand{\MM}{\ensuremath{\v{M}}}
\newcommand{\Mt}{\ensuremath{\v{M}_{\bot}}}
\newcommand{\BB}{\ensuremath{\v{B}}}
\newcommand{\Bo}{\ensuremath{\v{B_0}}}
\newcommand{\rr}{\ensuremath{R(\v{x})}}
\newcommand{\ww}{\ensuremath{\omega(\v{x})}}
\newcommand{\uu}{\ensuremath{\v{u}}}


%% MISC
\newcommand{\ab}[1]{\left| #1 \right|} % for absolute value
\newcommand{\avg}[1]{\left< #1 \right>} % for average
\let\underdot=\d % rename builtin command \d{} to \underdot{}
\let\baraccent=\= % rename builtin command \= to \baraccent
\renewcommand{\=}[1]{\stackrel{#1}{=}} % for putting numbers above =
\providecommand{\fr}{\frac}
\providecommand{\RR}{\mathbb{R}}
\providecommand{\CC}{\mathbb{C}}
\providecommand{\NN}{\mathbb{N}}
\providecommand{\e}{\epsilon}
\DeclareMathOperator{\di}{d\!}
\DeclareMathOperator{\Arg}{Arg}
\newcommand*\ieval[3]{\left.#1\right\rvert_{#2}^{#3}}

%% Vectors
\renewcommand{\v}[1]{\ensuremath{\mathbf{#1}}} 
\newcommand{\gv}[1]{\ensuremath{\mbox{\boldmath$ #1 $}}} % for vectors of Greek letters
\newcommand{\uv}[1]{\ensuremath{\mathbf{\hat{#1}}}} % for unit vector
\providecommand{\wave}[1]{\v{\tilde{#1}}}

%% DERIVATIVES
\renewcommand{\d}[2]{\frac{d #1}{d #2}} % for derivatives
\newcommand{\dubd}[2]{\frac{d^2 #1}{d #2^2}} % for double derivatives
\newcommand{\pd}[2]{\frac{\partial #1}{\partial #2}} % for partial derivatives
\newcommand{\ddt}[1]{\ensuremath{{#1}_t}}
\newcommand{\pdd}[2]{\frac{\partial^2 #1}{\partial #2^2}} % for double partial derivatives
\newcommand{\dx}[1]{\ensuremath{\operatorname{d}\!{#1}}}

%% Operators
\newcommand{\Gradient}{\ensuremath{\mbox{$\nabla$}}} %\newcommand{\Gradient}{\ensuremath{\mbox{\boldmath$\nabla$}}} % gradient
\newcommand{\Laplacian}{\ensuremath{\mbox{$\Delta$}}} % laplacian
%\newcommand{\Lap}{\nabla^2}
%\newcommand{\conv}{\ensuremath{\mbox{\ast}}} % convolution
\newcommand{\conv}{\ast} % convolution

%% Norms
\newcommand{\Norm}[1]{\ensuremath{\left\lVert #1 \right\rVert}} % generic norm
\newcommand{\ANorm}[2]{\ensuremath{\left\lVert #1 \right\rVert_{#2}}} % norm with respect to specified A
\newcommand{\LTwoNorm}[1]{\ensuremath{\left\lVert #1 \right\rVert_{L^2}}} % L2 norm
\newcommand{\HOneNorm}[1]{\ensuremath{\left\lVert #1 \right\rVert_{H^1}}} % H1 norm

%% Math Operators
\DeclareMathOperator{\spann}{span}

% ***************************************************
% ******************* TEXT AND CODE *****************
% ***************************************************

%% Text
\newcommand{\mathcolorbox}[2]{\colorbox{#1}{$\displaystyle #2$}}
\newcommand{\hlfancy}[3]{\textcolor{#1}{\sethlcolor{#2}\hl{#3}}}
\newcommand{\TODO}[1]{\hlfancy{red}{yellow}{\textbf{TODO: #1}}}

%% Code
%\newcommand{\code}[1]{\mintinline{C}{#1}}
%\newcommand{\code}[1]{\texttt{#1}}
%\newcommand{\code}[1]{\lstinline[columns=fixed]{#1}}
%\newcommand{\includecode}[1]{\lstinputlisting{#1}}

% ***************************************************
% **************** SPIN-ECHO COMMANDS ***************
% ***************************************************

% Remove algorithm number label, as there is only one
%\captionsetup[algorithm]{labelformat=empty}

%\newcommand{\vect[1]}{\vec{#1}} %vectors using arrows
%\newcommand{\vect}[1]{\mathbf{#1}} %vectors using boldface
%\newcommand{\bvect}[1]{\bm{#1}} % \mathbf doesn't work for greek/mathcal letters
%\newcommand{\bvect}[1]{#1} % looks better without bold

\newcommand{\matr}[1]{\mathbf{#1}} % matrix: undergraduate algebra version
%\newcommand{\matr}[1]{\bm{#1}}    % matrix: ISO complying version

%\newcommand{\tensor}[1]{\overleftrightarrow{\matr{#1}}} % for diffusion tensor
%\newcommand{\tensor}[1]{\overset{\text{\tiny$\leftrightarrow$}}{\matr{#1}}}
\newcommand{\tensor}[1]{\matr{#1}} % for diffusion tensor
%\def\tensor#1{\protect\@ontopof{#1}{\leftrightarrow}{1.15}\mathord{\box2}}

% for capital 'M' for molar
%\DeclareSIUnit\Molar{\text{M}}

%% Prevent inline equations from breaking over lines
%\relpenalty=9999 %10000 means never, 9999 means almost never
%\binoppenalty=9999
%
%% Prevent over hyphenation of words across lines
%\pretolerance=2000 %10000 means never, 9999 means almost never
%\tolerance=2000 
%\emergencystretch=10pt % maximum width of allowed extra space is 10pt

%% Begin added or changed by Alex Rauscher
\renewcommand{\floatpagefraction}{0.8} % so that pictures don't occupy an entire column

% force figures to be at the top of the page, not float in the middle
\makeatletter
\setlength{\@fptop}{0pt}
\makeatother

%\DeclareUnicodeCharacter{01E7}{ǧ} % does NOT work; pdflatex halts
%\newunicodechar{ǧ}{\u{g}} % not sure why, but this needs to be defined

% For a regular tilde...
\newcommand{\mytilde}{\raise.17ex\hbox{$\scriptstyle\mathtt{\sim}$}}
\newcommand{\myast}{\lower.17ex\hbox{*}}


\begin{document}

\begin{frame}
\titlepage
\begin{textblock*}{2cm}(1.3cm,-3.2cm)
    \includegraphics[height=2cm]{figures/ubc-logo}
\end{textblock*}
\begin{textblock*}{2cm}(9cm,-3.2cm)
    \includegraphics[height=2cm]{figures/UBC-MRI-logo}
\end{textblock*}
\end{frame}

\begin{frame}
\tableofcontents
%TODO
\TODO{uncomment all ``pause'' statements}
\end{frame}


\section{Background}

\subsection{Magnetic Resonance Visualised}

% Visualisation of the bulk magnetisation. Initially, all magnetised spins are aligned with the main magnetic field. 90 deg RF pulse kicks them into the xy-plane, and they begin to dephase. Note: this is RELATIVE spin rates; all spins are rotating in nearly the same direction, in fact, the relative rate between them differs by only ~1 part in 10^6
\begin{frame}
\frametitle{Magnetic Resonance Visualised}
\begin{enumerate}
    \item Animation of a typical ``spin echo'' MRI sequence
    \item \TODO{re-compile with animated gif} %TODO
\end{enumerate}
\centering
%\animategraphics[loop,controls,width=0.7\linewidth]{25}{gifs/HahnEcho_GWM-}{0}{255}
\includegraphics[width=0.7\linewidth]{gifs/HahnEcho_GWM-100}
\end{frame}

\begin{frame}
\frametitle{Magnetic Resonance Visualised}
\begin{enumerate}
    \item In actuality, spins (water molecules) do not truly fully refocus
    %\pause
    \item Relative angular frequency depends on local magnetic field, and therefore spins dephase at different rates at different locations
    %\pause
    \item In particular, the \textbf{diffusion} of spins during the scan ($\approx$ \SI{40}{\milli\second}) leads to a net lost in signal: \underline{the ``echo'' is weaker}
\end{enumerate}
\centering
\includegraphics[width=0.4\linewidth]{gifs/HahnEcho_GWM-165}
\end{frame}

\subsection{The Bloch-Torrey Equation}

\begin{frame}
\frametitle{The Bloch-Torrey Equation}
\begin{enumerate}
    \item We model this system through the \textbf{Bloch-Torrey equation}
    $$ \pd{\Mxy}{t} = D \Laplacian{\Mxy} - \CDecay \Mxy $$
    where:
    \begin{align*}
        \Mxy &= M_x + i M_y \\
        \CDecay(\v{x}) &= R(\v{x}) + i\omega(\v{x})
    \end{align*}
    %\pause
    \item The initial magnetization $\Mxy(\v{x},0) = \Mxy_0(\v{x})$ is given
    %\pause
    \item Boundary conditions are typically zero Neumann or periodic
\end{enumerate}
\end{frame}

\subsection{Problem Geometry}

\begin{frame}
\frametitle{Magnetic Resonance Visualised}
\begin{enumerate}
    \item Animation of a typical ``spin echo'' MRI sequence
    \item \TODO{re-compile with animated gif} %TODO
\end{enumerate}
\centering
%\animategraphics[loop,controls,width=0.7\linewidth]{25}{gifs/HahnEcho_GWM-}{0}{255}
\includegraphics[width=0.7\linewidth]{gifs/HahnEcho_GWM-100}
\end{frame}


\section{Solving the Bloch-Torrey Equation}


\subsection{Operator Splitting Methods}

\begin{frame}
\frametitle{Operator Splitting Methods}
\begin{enumerate}
    \item One effective method of solving the BT equation is via \textit{operator splitting methods}
    %\pause
    \item First, we re-write the BT PDE in the more suggestive form
    $$ \pd{\Mxy}{t} = H \Mxy $$
    where
    \begin{gather*}
        H = -D \Laplacian{} + \CDecay.
    \end{gather*}
    %\pause
    \item Then, the general solution $\Mxy$ may then be written as
    $$ \Mxy = e^{-Ht}\Mxy_0 $$
    where $e^{-Ht}$ is the \textit{evolution operator}
\end{enumerate}
\end{frame}

\begin{frame}
\frametitle{Operator Splitting Methods}
\begin{enumerate}
    \item Now, the evolution operator may be \textit{split} using the approximation %TODO ref
    \begin{align*}
        e^{-Ht} &= e^{D \Delta t - \CDecay t} \\
        &\approx e^{-\CDecay t/2} e^{D \Delta t} e^{-\CDecay t/2} + \mathcal{O}(t^3)
    \end{align*}
    %\pause
    %\item This form is advantageous because although $e^{-Ht}$ has no closed form for general $\CDecay$, the split operators do:
    \item Although $e^{-Ht}$ has no closed form, the split operators do:
    \begin{align*}
        e^{-\Gamma t/2} \Mxy &= e^{-\Gamma(\v{x}) t/2} \odot \Mxy \\
        e^{D \Delta t} \Mxy &= \Phi \conv \Mxy
    \end{align*}
    where $\odot$ is the Hadamard (pointwise) product, $\conv$ is the spatial convolution, and $\Phi$ is a Gaussian smoothing kernel with $\sigma = \sqrt{2Dt}$
\end{enumerate}
\end{frame}

\subsection{Finite Element Methods}

\begin{frame}
\frametitle{Finite Element Methods}
\begin{enumerate}
    \item The BT equation can also be solved using FEM
    %\pause
    \item First, we let $u=M_x$ and $v=M_y$ and re-write the complex Bloch-Torrey PDE as a pair of coupled real PDE's:
    \begin{align*}
        \pd{u}{t} &= D \Laplacian{u} - R u + \omega v \\
        \pd{v}{t} &= D \Laplacian{v} - R v - \omega u
    \end{align*}
    with
    \begin{align*}
        u(\v{x},0) &= M_x(\v{x},0) \\
        v(\v{x},0) &= M_y(\v{x},0)
    \end{align*}
\end{enumerate}
\end{frame}

\begin{frame}
\frametitle{Finite Element Methods}
\begin{enumerate}
    \item Applying the method of lines, the pair of PDE's becomes
    \begin{align*}
        M^h \v{u}_t &= -(D K^h + R^h) \v{u} + W^h \v{v} \\
        M^h \v{v}_t &= -(D K^h + R^h) \v{v} - W^h \v{u}
    \end{align*}
    where $R^h_{ij} \coloneqq \int R \,\phi_i \,\phi_j \dx{x}$, $W^h_{ij} \coloneqq \int \omega \,\phi_i \,\phi_j \dx{x}$, and $M^h$ and $K^h$ are the usual mass and stiffness matrices
    %\pause
    \item $M^h$, $K^h$, and $R^h$ are symmetric positive definite; $W^h$ is symmetric
    %\pause
    \item In choosing a time discretisation, first consider the block system:
    \begin{align*}
    \begin{pmatrix}
    M^h & 0 \\ 
    0 & M^h
    \end{pmatrix}
    \pd{}{t}
    \begin{pmatrix} \v{u} \\ \v{v} \end{pmatrix}
    = -
    \begin{pmatrix}
    A^h & -W^h \\ 
    W^h & A^h
    \end{pmatrix}
    \begin{pmatrix} \v{u} \\ \v{v} \end{pmatrix}
    \end{align*}
    \\ where $A^h \coloneqq D K^h + R^h$
\end{enumerate}
\end{frame}

\begin{frame}
\frametitle{Finite Element Methods}
\vspace{-1cm}
\begin{align*}
    \begin{pmatrix}
    M^h & 0 \\ 
    0 & M^h
    \end{pmatrix}
    \pd{}{t}
    \begin{pmatrix} \v{u} \\ \v{v} \end{pmatrix}
    = -
    \begin{pmatrix}
    A^h & -W^h \\ 
    W^h & A^h
    \end{pmatrix}
    \begin{pmatrix} \v{u} \\ \v{v} \end{pmatrix}
\end{align*}
\begin{enumerate}
    \item If we take the inner product of both sides with $\begin{pmatrix} \v{u}^T, \v{v}^T \end{pmatrix}$, we have that
    \begin{align*}
    \pd{}{t}
    \left( \v{u}^T M^h \v{u} + \v{v}^T M^h \v{v} \right)
    %&= - \left( \v{u}^T A^h \v{u} - \v{u}^T W^h \v{v} + \v{v}^T W^h \v{u} + \v{v}^T A^h \v{v} \right) \\
    &= - \left( \v{u}^T A^h \v{u} + \v{v}^T A^h \v{v} \right)
    \quad \text{by symmetry of } W^h \\
    &\leq 0 \quad \text{by positive definiteness of } A^h
    \end{align*}
    %\pause
    \item Now, $\LTwoNorm{\bm{\Mxy}}^2 = \v{u}^T M^h \v{u} + \v{v}^T M^h \v{v}$, and so $\LTwoNorm{\bm{\Mxy}}$ decreases with time
    %\pause
    \item For this reason, the second order and strongly A-stable time stepping \textsc{TR-BDF2} was used
\end{enumerate}
\end{frame}


\section{Results}

\subsection{Operator Splitting vs. FEM}

\end{document}